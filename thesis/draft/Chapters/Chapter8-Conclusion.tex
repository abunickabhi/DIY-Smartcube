\chapter{Conclusion} 
\label{Chapter8}

The D.I.Y. Smartcube seeks to enable speedcubers to obtain the same
metrics and data from their main speedcube that are offered by existing
commercial smartcubes. This chapter summarizes the progress made toward
that goal (Section \ref{sec:answering-research-questions}), the hurdles
that remain (Section \ref{sec:limitations}), and the next milestones
that await future research (Section \ref{sec:outlook}).


\section{Answering Research Questions}
\label{sec:answering-research-questions}

Chapter \ref{Chapter3} concluded with a detailed list of research
questions that would be explored in this thesis (see Section
\ref{sec:research-questions}). This section provides direct, summarized
answers to each of those questions.

Section \ref{subsec:answer-feasibility} summarizes the constraints for
a sound-based move tracking protocol. Section
\ref{subsec:answer-accuracy} then reports on the accuracy achieved by
the proof-of-concept proposed earlier in the thesis. Section
\ref{subsec:answer-compatibility} continues by reviewing the challenges
associated with building a transmitter small enough to
non-destructively embed into an existing speedcube. From there, Section
\ref{subsec:answer-granularity} discusses additional data points that
can be inferred from the transmitted audio. Finally, Section
\ref{subsec:answer-competition-legality} addresses the ways in which a
sound-based smartcube design can comply with competition regulations
banning the usage of electronics.

\subsection{Feasibility on Consumer Hardware}
\label{subsec:answer-feasibility}

\emph{What are the constraints for a sound-based smartcube design
compatible with consumer-grade microphones like those found in common
smartphones and laptops?}

As discussed in Chapter \ref{Chapter4}, there are four main factors
that affect the viability of a sound-based smartcube design: 

\begin{enumerate}
    \item the achievable \textbf{signal-to-noise ratio} of the transmitter
    \item the \textbf{distinctiveness of the tones} used for the protocol
    \item the \textbf{range of tones audible} to the human ear
    \item the \textbf{frequency response range of consumer hardware} used to implement the protocol
\end{enumerate}

An analysis of these factors found that the tones used in the protocol
should be (1) broadcast with an audio intensity of -30dB or greater
(see Section \ref{subsec:signal-to-noise-ratio}) and (2) spaced at
least 100Hz apart (see Section \ref{subsec:tone-distinctiveness}) in
order to have a sufficiently strong signal-to-noise ratio for clear
detection. Furthermore, the specific tones chosen for the protocol
would ideally fit within the "musical range" of 500Hz and 4000Hz to
simultaneously (3) avoid irritating human solvers (see Section
\ref{subsec:human-auditory-range}) and (4) leverage the optimizations
built into consumer audio recording hardware in smartphones and laptops
(see Section \ref{subsec:frequency-response-range}).

\subsection{Move Tracking Accuracy}
\label{subsec:answer-accuracy}

\emph{How could a sound-based smartcube design track the face turns of
a Rubik's Cube with high accuracy?}

As discussed in Section \ref{sec:move-tracking-accuracy}, the receiver
from Chapter \ref{Chapter5} and the transmitter from Chapter
\ref{Chapter6} were able to transmit and decode a sequence of face
turns with perfect accuracy across a variety of configuration
parameters within a controlled environment. \footnote{Specifically,
this controlled environment consisted of synthetically generated audio
representative of the output of an ideal transmitter (see Sections
\ref{sec:synthetic-audio-generation} and
\ref{sec:adding-realistic-noise}) played from a Google Pixel phone and
recorded by an HP Spectre laptop in a quiet room (see Section
\ref{subsec:signal-to-noise-ratio}).}

Various factors aided in these results. First, each face position is
associated with a signal present in a unique frequency band (See
Section \ref{subsec:tone-distinctiveness}). This separation of
frequencies makes it easier to distinguish changes in the state of
independent faces on the cube. Second, the current position of each
face is continuously broadcast, increasing the amount of time available
for the receiver to detect and decode each position over time (See
Section \ref{sec:alternatives}). Third, multiple layers of noise
filters were applied within the receiver to better focus on the true
signal frequencies and boost confidence in the accuracy of the decoded
face turns (See Section \ref{sec:decoding-realistic-noise}).


\subsection{Compatibility with Standard Speedcubes}
\label{subsec:answer-compatibility}

\emph{How could a sound-based smartcube design be deployed within a
standard, "non-smart" speedcube without requiring permanent
modifications to the original cube?}

The key challenges involved in creating a sound-based smartcube
enhancement for an existing speedcube stem from the size constraints
imposed by the structure of the cube. Since all modifications must be
non-destructive, the required electronics must be embedded into the
empty spaces within the cubies on the cube. Furthermore, since neither
the edges nor the corners on the cube have access to the axles on which
the cube rotates, the best place for these electronics is within the
centercaps themselves. Unfortunately, these centercaps offer little
space for add-ins. For example, as discussed in Sections
\ref{subsec:prospects-of-miniaturization} and
\ref{sec:miniaturization}, the relatively spacious Gans 356 only offers
space for a $16mm^2$ PCB with rounded corners and a $6mm$ diameter hole
in the middle.

Despite these stringent size constraints, Chapter \ref{Chapter6}
specifies a promising transmitter design consisting of less than 10
discrete electrical components. With this limited component count, a
PCB consisting of SMD components would be able to fit within a
custom-printed centercap for a Gans 356 speedcube as shown in Figure
\ref{fig:core-placement}.\footnote{Note that an actual PCB was not
constructed nor was a PCB layout attempted for any other cube than the
Gans 356.}

As discussed in Section
\ref{sec:compatibility-with-standard-speedcubes}, since the required
electronics are contained within a replaceable, custom centercap, the
modifications can be easily reversed by replacing the custom centercaps
with the originals.


\subsection{Move Tracking Granularity}
\label{subsec:answer-granularity}

\emph{How could a sound-based smartcube design record the time spent
executing each individual face turn of a Rubik's Cube?}

The receiver proposed in Chapter \ref{Chapter5} decodes face turns by
iterating over a time series of audio data. As a result, when a face
turn is detected, the receiver also knows the time at which it
occurred. By computing the difference between the times at which two
successive turns occurred, the time spent executing an individual face
turn can be computed. Thus, as long as the receiver can detect the face
turns applied to the cube with high accuracy, then the time spent
executing each one can be easily calculated (See also Section
\ref{sec:move-tracking-granularity}).


\subsection{Competition Legality}
\label{subsec:answer-competition-legality}

\emph{How could a sound-based smartcube design comply with competition
regulations prohibiting the use of electronics while performing a
competitive solve?}

Since the use of electronics while competitively solving a Rubik's Cube
is banned by WCA regulation 2i \cite{wca-regulations}, any move
tracking solutions must not permanently embed electronics into the
structure of the cube.

The solution proposed in Section
\ref{subsec:prospects-of-miniaturization} is to create a tiny sound
transmitter that can be embedded into a custom centercap. Such a
transmitter can be temporarily installed during personal practice then
replaced with the original centercaps during competitions.

Alternatively, since the signal of a sound-based transmitter can be
blocked by by a physical wall or purposefully broadcasting interfering
signals in the same frequency range, it's possible that, with proper
negotiation with the WCA, a sound based move tracking solution could be
deemed competition legal.


\section{Limitations}
\label{sec:limitations}

As stated in Section \ref{sec:research-questions}, the overarching goal
of this thesis is to present a proof-of-concept for a sound-based smart
cube design. Like all proofs-of-concept, the design proposed in this
thesis contains some inherent flaws and underlying assumptions that limit
its ability to be successfully deployed in existing speedcubes.

First, while Chapter \ref{Chapter6} demonstrates that its proposed
transmitter design appropriately considers all relevant requirements, a
fully functional transmitter was never built and deployed within an
existing speedcube. As such, the viability of this transmitter design
cannot be fully verified until a transmitter is fully constructed,
deployed, and tested.

Second, while great care was taken to extensively test the receiver,
all tests were conducted with realistic, but synthetic audio samples
since a physical transmitter was never built (See Section
\ref{sec:adding-realistic-noise}). As such, it is possible that further
refinements to the receiver will be required if the signals from a real
transmitter substantially differ from those synthesized for testing the
receiver.

Third, though not a functional limitation, since the typical frequency
response range of consumer hardware is entirely within the human
auditory range (see Sections \ref{subsec:frequency-response-range} and
\ref{subsec:human-auditory-range}), all frequencies selected for
transmitting the state of the cube will be human-audible. If the chosen
frequencies are higher in pitch (as recommended by Section
\ref{subsec:freq-selection}), the sounds produced by the transmitter
may prove irritating to the speedcuber. 

Fourth and finally, while the tests of the receiver achieved perfect
accuracy on all test cases (see Section \ref{subsec:answer-accuracy}),
the specific parameters used to achieve perfect accuracy differed
between each one. Since there is not one set of parameters that worked
for all test cases, additional experimentation and refinement will be
necessary to determine the best parameters for decoding face turns in
any given cube/environment combination.


\section{Outlook}
\label{sec:outlook}

The proof-of-concept shown in this thesis demonstrates the potential of
a sound-based move tracking system for realizing the goal of the D.I.Y.
Smartcube: to enable speedcubers to obtain the same metrics and data
from their main speedcube that are offered by existing commercial
smartcubes.

To address the specific issues discussed in the Limitations section
above (\ref{sec:limitations}), the following areas could be studied
more thoroughly:

\begin{enumerate}

    \item \textbf{Transmitter Construction}: Designing, manufacturing,
    and testing the actual PCB that would be embedded into a Rubik's
    Cube's centercap. This may include creating variations of the PCB
    for different types of speedcubes.

    \item \textbf{Pairing the Receiver with a Physical Transmitter}: Once
    a physical transmitter is constructed, re-run the battery of tests
    described in Section \ref{sec:move-tracking-accuracy} to assess the
    receiver's actual accuracy with a physical transmitter.

    \item \textbf{Supersonic Frequencies}: Explore the viability of
    moving the frequency bands used for face turn transmission into the
    supersonic range (>20kHz) so that they cannot distract human
    speedsolvers. At a minimum, this would require connecting a
    supersonic microphone into a laptop or smartphone to detect the
    higher frequency tones.

    \item \textbf{Refined Receiver Parameters}: Create a collection of
    optimal receiver parameters for a variety of combinations of cubes
    and environments. Additional research could also explore ways to
    determine these optimal parameters dynamically as a speedcuber
    gives feedback on the quality of the face turn decoding.

\end{enumerate}

There are also a few areas other of potential research that would
enhance the experience of using a sound-based smartcube.

\begin{itemize}

    \item \textbf{Live Algorithm}: The current implementation of the
    receiver records the entire move sequence, then processes it all at
    once. A more interactive version would be able to continuously
    listen to the microphone and detect face turns in real time.

    \item \textbf{Support for Cube Rotations}: As mentioned in Section
    \ref{subsec:represent-audio-protocol}, the set of decoded face
    turns assumes no cube rotations. Cube rotations are extremely
    common during speedsolves and pose interesting challenges for move
    tracking systems since they do not rotate any axles but do change
    which color centerpiece is associated with each face (see Section
    \ref{sec:rubiks-anatomy}). Potential solutions could involve the
    addition of a hardware gyroscope and/or a set of software
    heuristics based on speedcubers' preferences for R U L D moves.

\end{itemize}

Additional support for future research in this area can also be found
by contacting the author and/or participating in speedsolving
communities such as
\href{https://www.speedsolving.com/}{Speedsolving.com}.