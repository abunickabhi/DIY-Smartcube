% Chapter Template

\chapter{Conclusion} % Main chapter title

\label{Chapter8} % Change X to a consecutive number; for referencing this chapter elsewhere, use \ref{ChapterX}


\section{Summary}

TODO summarize the goals from the Introduction and broadly describe
what techniques were most helpful in reaching them. Also detail a
summary of exactly how effective they were.

\section{Answering Research Questions}
\label{sec:answering-research-questions}

\subsection{Feasibility on Consumer Hardware}
\label{subsec:answer-feasibility}

\emph{What are the constraints for a sound-based smart cube design
compatible with consumer-grade microphones like those found in common
smartphones and laptops?}

As discussed in Chapter \ref{Chapter4}, there are four main factors
that affect the viability of a sound-based smart cube design: the
achievable signal-to-noise ratio of the transmitter, the
distinctiveness of the tones used for the protocol, the frequency
response range of the consumer hardware used to implement the protocol,
and the range of tones audible to the human ear.

An analysis of these factors found that the tones used in the protocol
should be spaced at least 100Hz apart and broadcast with an audio
intensity of -30dB or greater in order to have a sufficiently strong
signal-to-noise ratio for clear detection. Furthermore, the specific
tones chosen for the protocol would ideally fit within the "musical
range" of 500Hz and 4000Hz to simultaneously avoid irritating human
solvers and leverage the optimizations built into consumer audio
recording hardware in smartphones and laptops.

\subsection{Move Tracking Accuracy}
\label{subsec:answer-accuracy}

\emph{How could a sound-based smart cube design track the face turns of
a Rubik's Cube with high accuracy?}

As discussed in Section \ref{sec:move-tracking-accuracy}, the receiver
from Chapter \ref{Chapter5} and the transmitter from Chapter
\ref{Chapter6} were able to transmit and decode a sequence of face
turns with perfect accuracy across a variety of input parameters within
a controlled environment. \footnote{Specifically, this controlled
environment consisted of synthetically generated audio representative
of the output of an ideal transmitter (see Sections
\ref{sec:synthetic-audio-generation} and
\ref{sec:adding-realistic-noise}) played from a Google Pixel phone and
recorded by an HP Spectre laptop in a quiet room (see Section
\ref{subsec:signal-to-noise-ratio}).}

Various factors aided in these results. First, each face position is
associated with a signal present in a unique frequency band (See
Section \ref{subsec:tone-distinctiveness}). This separation of
frequencies makes it easier to distinguish changes in the state of
independent faces on the cube. Second, the current position of each
face is continuously broadcast, increasing the amount of time available
for the receiver to detect and decode each position over time (See
Section \ref{sec:alternatives}). Third, multiple layers of noise
reductions were applied within the receiver to better focus on the true
signal frequencies and boost confidence in the accuracy of the decoded
face turns (See Section \ref{sec:decoding-realistic-noise}).


\subsection{Compatibility with Standard Speedcubes}
\label{subsec:answer-compatibility}

\emph{How could a sound-based smart cube design be deployed within a
standard, "non-smart" speedcube without requiring permanent
modifications to the original cube?}

The key challenges involved in creating a sound-based smartcube
enhancement for an existing speedcube stem from the size constraints
imposed by the structure of the cube. For example, as discussed in
Sections \ref{subsec:prospects-of-miniaturization} and
\ref{sec:miniaturization}, the particularly spacious Gans 356 only
offers space for a 16mm$^2$ PCB with rounded corners and a
\textbf{SIZE} hole in the middle.

Section \ref{sec:miniaturization} lays out a sample prototype indicating ... TODO continue here!
% TODO Make sure to not answer this with 100% certainty since we didn't build a fully functioning prototype. Also include in *outlook*.

\subsection{Move Tracking Granularity}
\label{subsec:answer-granularity}

\emph{How could a sound-based smart cube design record the time spent
executing each individual face turn of a Rubik's Cube?}

The receiver proposed in Chapter \ref{Chapter5} decodes face turns by
iterating over a time series of audio data. As a result, when a face
turn is detected, the receiver also knows the time at which it
occurred. By computing the difference between the times at which two
successive turns occurred, the time spent executing an individual face
turn can be computed. Thus, as long as the receiver can detect the face
turns applied to the cube with high accuracy, then the time spent
executing each one can be easily calculated (See also Section
\ref{sec:move-tracking-granularity}).


\subsection{Competition Legality}
\label{subsec:answer-competition-legality}

\emph{How could a sound-based smartcube design comply with competition
regulations prohibiting the use of electronics while performing a
competitive solve?}

Since the use of electronics while competitively solving a Rubik's Cube
is banned by WCA regulation 2i \cite{wca-regulations}, any move
tracking solutions must not permanently embed electronics into the
structure of the cube.

The solution proposed in Section
\ref{subsec:prospects-of-miniaturization} is to create a tiny sound
transmitter that can be embedded into a custom centercap. Such a
transmitter can be temporarily installed during personal practice then
replaced with the original centercaps during competitions.

Alternatively, since the signal of a sound-based transmitter can be
blocked by by a physical wall or purposefully broadcasting interfering
signals in the same frequency range, it's possible that, with proper
negotiation with the WCA, a sound based move tracking solution could be
deemed competition legal.


\section{Limitations}

TODO detail the limits to which this research can be more broadly
applied. Be clear about the effects various assumptions/decisions have
on the reliability of the conclusions of this thesis.

- Contradiction of a "silent protocol" with consumer hardware -> higher frequency transmission + high-res mic to plug into device.
- There were perfect parameters for each scramble, but how do we know which parameters to use for an arbitrary scramble?


\section{Outlook}

TODO If I had more time/resources to work on it, what would I do next
with this project?

- Live algorithm
- build it small enough to fit in the cube.