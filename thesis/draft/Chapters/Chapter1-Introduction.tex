\chapter{Introduction} 
\label{Chapter1}


\section{The Advent of Smart Cubes}

Speedsolving, the sport of solving twisty puzzles like the Rubik's Cube
as fast as possible, has seen a resurgence of popularity since the
early 2000s. \cite{wca-competition-history} Over the past two decades
many advances in cube technology have produced ever higher performing
puzzles.

Recently, the speedcubing community has seen the entrance of smart
cubes, special versions of a Rubik's Cube built around hardware that
can connect to a mobile device over Bluetooth. These smart cubes have
sparked a wave of excitement with the vast opportunities they offer for
automatic turn tracking, performance analysis, personalized improvement
feedback, and networked competition.


\section{Obstacles to Adoption}

While a revolutionary idea, smart cubes still face several obstacles to
widespread adoption.

\begin{itemize}

    \item \emph{Cost}: Smart cubes generally cost two to eight times as
    much as a comparable, non-smart speedcube. \footnote{For example,
    one popular budget speedcube, the Yuxin Little Magic, costs only
    \$5 \cite{yuxin-thecubicle}, while the cheapest smartcube, the
    Giiker Cube, starts at \$42 \cite{giiker-thecubicle}. On the higher
    end, a premium speedcube, like the Gans 356 XS, retails for \$53
    \cite{gans-xs-thecubicle} while a premium smartcube, like the
    GoCube, retails just under \$100. \cite{gocube-price}}
    
    \item \emph{Performance}: Existing smart cubes turn slower than
    comparable non-smart cubes, making them less reflective of a
    speedcuber's true skill. \cite{smartcube-regulation-discussion}
    
    \item \emph{Reliability}: Many smart cube owners report inability
    to connect the smart cube to a mobile device and missed/inaccurate
    turn tracking. \cite{smartcube-regulation-discussion}
    
    \item \emph{Regulation}: Current competition rules ban the use of
    electronics during timed solves, thus banning the use of smart
    cubes \cite{wca-regulations}. There is no foreseeable change to
    this rule. \cite{smartcube-regulation-discussion}
    
\end{itemize}

As a result of these obstacles, many speedcubers refrain from
purchasing a smart cube, despite expressing significant interest in the
opportunities smart cubes offer.

Furthermore, all current smartcubes have been specifically built for
the primary purpose of providing move-tracking functionality. There is
no existing way to automatically track the moves of a standard,
"non-smart" speedcube.


\section{Purpose of this Thesis}
\label{sec:thesis-purpose}

The primary goal of this thesis is to create a proof-of-concept for a
smart cube design that can enable speedcubers to obtain the same
metrics and data that are offered by existing commercial smartcubes
from their main, non-smart speedcube.

In other words, this thesis will explore the following question in depth:

\emph{Is it possible to track the face turns of a standard, "non-smart"
speedcube in a non-destructive, competition-legal way?}


\section{Thesis Overview}
\label{sec:thesis-overview}

This exploration begins in Chapter \ref{Chapter2} with an introduction
of several foundational topics about the structure of a Rubik's Cube
along with an overview of the speedcubing community. Chapter
\ref{Chapter3} reviews relevant efforts in the field of smartcube
construction and outline the detailed research questions for this
thesis. Chapter \ref{Chapter4} then defines and analyzes the various
factors to consider when building a sound-based smartcube. Chapter
\ref{Chapter5} begins the proof-of-concept design by proposing a
software receiver for a sound-based move tracking protocol. Chapter
\ref{Chapter6} completes the proof-of-concept by proposing a design for
the complementary physical transmitter to embed into a speedcube.
Chapter \ref{Chapter7} then evaluates the performance of the receiver
and transmitter against a variety of key benchmarks. Finally, Chapter
\ref{Chapter8} synthesizes answers to the specific research questions
from Chapter \ref{Chapter3} and discusses the outlook for future
research in the domain of sound-based Rubik's Cube turn tracking.
