% Chapter Template

\chapter{Introduction} % Main chapter title

\label{Chapter1} % Change X to a consecutive number; for referencing this chapter elsewhere, use \ref{ChapterX}

% TODO What are we looking at? What will be shown? State general research questions to answer. Start by defining the *problem*. The rest of the thesis will detail my solution.

% I am trying to persuade someone that a sound-based smart cube is viable on consumer grade hardware.

\section{The Advent of Smart Cubes}
Speedsolving, the sport of solving twisty puzzles like the Rubik’s Cube as fast as possible, has seen a resurgence of popularity since the early 2000s. % TODO add citation 
Over the past two decades many advances in cube technology have produced ever higher performing puzzles. 

Recently, the speedcubing community has seen the entrance of smart cubes, special versions of a Rubik’s Cube that can connect to a mobile device over Bluetooth. 
These smart cubes have sparked a wave of excitement with the vast opportunities they offer for automatic turn tracking, performance analysis, personalized improvement feedback, and networked competition.

\section{Obstacles to Adoption}
While a revolutionary idea, smart cubes still face several obstacles to widespread adoption. 

\begin{itemize}
    \item \emph{Cost}: Smart cubes can cost up to eight times as much as a comparable non-smart speedcube. For example, one popular budget speedcube, the Moyu Weilong, costs only \$5, while the cheapest smartcube, the Giiker Cube, starts at \$40. [TODO sources!]. On the higher end, a premium speedcube, like the Gans 356 XS, retails for just over \$60 while a premium smartcube, like the GoCube, retails for over \$100. [TODO Make this a footnote.]
    \item \emph{Performance}: Existing smart cubes turn slower than comparable non-smart cubes. [TODO source]
    \item \emph{Reliability}: Many smart cube owners report inability to connect the smart cube to a mobile device and missed/inaccurate turn tracking.
    \item \emph{Regulation}: Current competition rules ban the use of electronics during timed solves, thus banning the use of smart cubes. There is no foreseeable change to this rule. 
\end{itemize}

As a result of these obstacles, many speedcubers refrain from purchasing a smart cube, despite expressing significant interest in the opportunities smart cubes offer.

\section{Purpose of this Thesis}
The primary goal of this thesis is to create a proof-of-concept for a smart cube design that mitigates the performance and competitive regulatory concerns of existing smartcubes.

\subsection{Design Requirements}
TODO 

Specifically, the smart cube design discussed in this thesis will be assessed against the following criteria:
\begin{enumerate}
    \item \emph{Compatibility with Standard Speedcubes}: The design must be deployable within a standard (non-smart) speedcube. "How can a standard speedcube be enhanced into a smart cube?"
        \begin{enumerate} 
            \item The design must not require permanent modifications to the original speedcube.
            \item The design must not significantly impact the turn-speed of the original speedcube.
        \end{enumerate}
    \item \emph{Move Tracking Accuracy}: The design must be capable of tracking the face turns of a Rubik's Cube with over 99\% accuracy. 
    \item \emph{Move Tracking Granularity}: The design must be capable of recording the time spent executing each individual face turn of a Rubik's Cube.
    \item \emph{Competition Legality}: The design must be competition legal, meaning it results in a cube that either does not violate existing competition rules against the use of electronics or can be easily modified to regain compliance.
\end{enumerate}

\section{Thesis Overview}
TODO give an overview of the rest of the Thesis document.