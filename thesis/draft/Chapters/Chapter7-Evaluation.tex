% Chapter Template

\chapter{Evaluation} % Main chapter title

\label{Chapter7} % Change X to a consecutive number; for referencing this chapter elsewhere, use \ref{ChapterX}

TODO review the core goals outlined in the introduction, and methodically review how well my final prototypes stack up against those goals. This section serves to prove (with all the data) that the approach I've described in the previous chapters actually works.


\section{Compatibility with Standard Speedcubes}

TODO discuss how well my design meets this requirement from the Introduction

The design must be deployable within a standard (non-smart) speedcube.
\begin{enumerate} 
    \item The design must not require permanent modifications to the original speedcube.
    \item The design must not significantly impact the turn-speed of the original speedcube.
\end{enumerate}


\section{Move Tracking Accuracy}

TODO discuss how well my design meets this requirement from the Introduction

The design must be capable of tracking the face turns of a Rubik's Cube with over 99\% accuracy.


\section{Move Tracking Granularity}

TODO discuss how well my design meets this requirement from the Introduction

The design must be capable of recording the time spent executing each individual face turn of a Rubik's Cube.


\section{Competition Legality}

TODO discuss how well my design meets this requirement from the Introduction

The design must be competition legal, meaning it results in a cube that either does not violate existing competition rules against the use of electronics or can be easily modified to regain compliance.

