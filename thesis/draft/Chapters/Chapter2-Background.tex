% Chapter Template

\chapter{Background} % Main chapter title

\label{Chapter2} % Change X to a consecutive number; for referencing this chapter elsewhere, use \ref{ChapterX}

TODO Each chapter starts with a paragraph that briefly outlines the purpose of each of the sections.


\section{A Brief History of the Rubik's Cube}
\label{sec:rubiks-history}
In 1974, Erno Rubik, a Hungarian professor of architecture, sought to help his students visualize space in three dimensions. 
To that end, he created a special cube whose faces could independently rotate around all three physical axes \cite{rubik-motivation}.
When he added colored stickers to further aid in visualizing the movements, Mr. Rubik realized he had created a new puzzle.
He patented his cube in 1975, \cite{rubik-patent} and since then over 450 million units have been sold \cite{forbes-rubik-merger}, allowing an estimated 1 in 7 humans on earth to try their hand at solving it \cite{rubik-population-reached}.

Since then, the cube has been the subject of academic research, competition, leisure, and cultural iconography.


\section{The Anatomy of a Rubik's Cube}
\label{sec:rubiks-anatomy}
The Rubik's Cube, like a standard geometric cube, has six faces, all of which are squares and positioned at right angles to each other.
When solved, each of these faces has a single, unique color.

The Rubik's Cube is further subdivided into a 3x3x3 arrangement of smaller "cubies" such that each face consists of nine individual colored stickers/tiles. 
There are three different types of cubies: centers, edges, and corners.
Each type of cubie is distinguished by the number of unique colors it binds together into a single physical unit.

\begin{table}[h]
    \centering
    \begin{minipage}{10cm}
        \begin{tabular}{ | c | c | c | }
            \hline
            Type of Cubie & Number of Colors & Count per Rubik's Cube \\ 
            \hline \hline
            Center & 1 & 6\footnote[1]{All six center pieces are attached a single core} \\
            \hline
            Edge & 2 & 12 \\  
            \hline
            Corner & 3 & 8 \\
            \hline
        \end{tabular}
    \end{minipage}
    \caption{The number of each type of Rubik's Cube cubie}
    \label{table:cubie-count}
\end{table}

Each of the six center cubies are also attached to a common core which allows them to rotate freely, but fixes their position relative to each other.
As such, the single color of each center cubie is also the color shared by the corresponding face when the entire cube is solved.

\subsection{Algorithm Notation}
TODO

\subsection{The Laws of the Cube}
TODO Describe the basic concepts of group theory that stipulate what positions are and aren't legal. It might also be fun to discuss the derivation of the 43 quintillion possible positions on the cube.


\section{Speedsolving}

TODO

\subsection{The World Cube Association}
TODO

\subsection{Competition Regulations}
TODO


\section{The Rise of Smart Cubes}

TODO