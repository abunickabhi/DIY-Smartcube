% Chapter Template

\chapter{Background} % Main chapter title

\label{Chapter2} % Change X to a consecutive number; for referencing this chapter elsewhere, use \ref{ChapterX}

TODO Each chapter starts with a paragraph that briefly outlines the purpose of each of the sections.


\section{A Brief History of the Rubik's Cube}

In 1974, Erno Rubik, a Hungarian professor of architecture, sought to help his students visualize space in three dimensions. 
To that end, he created a special cube whose faces could independently rotate around all three physical axes \cite{rubik-motivation}.
When he added colored stickers to further aid in visualizing the movements, Mr. Rubik realized he had created a new puzzle.
He patented his cube in 1975, \cite{rubik-patent} and since then over 450 million units have been sold \cite{forbes-rubik-merger}, allowing an estimated 1 in 7 humans on earth to try their hand at solving it \cite{rubik-population-reached}.

Since then, the cube has been the subject of academic research, competition, leisure, and cultural iconography.


\section{The Mechanics of the Rubik's Cube}



\subsection{Algorithm Notation}
TODO

\subsection{The Laws of the Cube}
TODO Describe the basic concepts of group theory that stipulate what positions are and aren't legal. It might also be fun to discuss the derivation of the 43 quintillion possible positions on the cube.


\section{Speedsolving}

TODO

\subsection{The World Cube Association}
TODO

\subsection{Competition Regulations}
TODO


\section{The Rise of Smart Cubes}

TODO