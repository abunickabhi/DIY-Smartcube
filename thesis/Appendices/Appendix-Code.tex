%%%%%%%%%%%%%%%%%%%%%%%%%%%%%%%%%%%%%%%%%
% The contents of this document are the copyright of Joseph Hale
%
% This Source Code Form is subject to the terms of the Mozilla Public
% License, v. 2.0. If a copy of the MPL was not distributed with this
% file, You can obtain one at https://mozilla.org/MPL/2.0/.
%%%%%%%%%%%%%%%%%%%%%%%%%%%%%%%%%%%%%%%%%

\chapter{Longer Code Snippets} % Main appendix title
\label{appendix:code} % For referencing this appendix elsewhere, use \ref{AppendixA}

\section{Spectrogram Generation for Figure \ref{fig:spectrogram}}
\label{sec:code-spectrogram}
This code snippet actually generates a full animation of how the component frequencies change over time.
Figure \ref{fig:spectrogram} is only one frame of this animation.

Make sure to run this code after running the code in Section \ref{subsec:compute-spectrogram}.
\begin{verbatim}
# Heavily aided by https://matplotlib.org/stable/gallery/animation/simple_anim.html
import matplotlib.pyplot as plt
import matplotlib.animation as animation

def compute_threshold(values: list):
    return -500  # Out of sight of the chart

def animate(wav_path: str, alg: str):
    # Parse the audio sample into its spectrogram
    freq, time, spectrogram = compute_spectrogram(wav_path)

    # Generate the plots
    fig, (ax_spectrogram, ax_fourier) = plt.subplots(2)

    # Set the spectrogram
    ax_spectrogram.specgram(audio_samples, Fs=sample_rate, 
        NFFT=SAMPLES_PER_WINDOW,
        noverlap=SAMPLES_PER_WINDOW // 4 * 3)  # 3/4 of the window size
    line_spectrogram, = ax_spectrogram.plot([], [], '-', color='red')
    time_tracker_spectrogram = ax_spectrogram.axvline(0, color='red')
    ax_spectrogram.set_ylim(0, 4000)
    ax_spectrogram.set_title(f"Spectrogram")
    ax_spectrogram.set_ylabel('Frequency [Hz]')
    ax_spectrogram.set_xlabel('Time [sec]')

    # Set the Fourier transform
    STEP = 10
    indices = np.arange(0, len(time) + 1, step=STEP)
    line_fourier, = ax_fourier.plot(freq, spectrogram[0], color='red')
    threshold_fourier = ax_fourier.axhline(-500, color='green')
    ax_fourier.set_title("")
    ax_fourier.set_xlim(0, 4000)  # Frequency
    ax_fourier.set_xlabel("Frequency [Hz]")
    ax_fourier.set_ylim(0, np.amax(spectrogram))  # Strength
    ax_fourier.set_ylabel("Strength")

    # Space figures nicely
    fig.suptitle(alg)
    fig.tight_layout(h_pad=3)

    def draw_frame_at(i):
        time_tracker_spectrogram.set_xdata(time[i])
        line_fourier.set_ydata(spectrogram[i])
        threshold_fourier.set_ydata(compute_threshold(spectrogram[i]))
        ax_fourier.set_title(f"Frequencies at {time[i]:.2f} seconds [idx={i}]")
        return line_spectrogram, line_fourier

    ani = animation.FuncAnimation(fig, draw_frame_at,
        frames=indices, interval=1, blit=True, save_count=len(time)/STEP)
    ani.save(f"plt/animation/{wav_path[:-4]}.mp4",
        fps=len(time)/STEP/5, dpi=300)
    plt.show()

animate(demo_wav_path, demo_alg)
\end{verbatim}


\section{Spectrogram Generation for Figure \ref{fig:spectrogram-with-naive-threshold}}
\label{sec:code-spectrogram-with-naive-threshold}
This code snippet actually generates a full animation of how the component frequencies change over time.
Figure \ref{fig:spectrogram-with-naive-threshold} is only one frame of this animation.

Make sure to run this code after running the code in Section \ref{subsec:extract-dominant-freqs} as the new \emph{compute\_threshold} function there is what adds the threshold line to this animation.

\begin{verbatim}
animate(demo_wav_path, demo_alg)
\end{verbatim}
